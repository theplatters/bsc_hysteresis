% !TeX program = xelatex
% !TeX encoding = UTF-8
% !TeX spellcheck = en_US
% !BIB program = biber
%%
%% The above lines help editors like TeXstudio to automatically choose the right tools
%% to compile your LaTeX source file. If your tool does not support these magic comments,
%% you will need to make appropriate manual choices.
%%
%% You can safely use "pdflatex" instead of "xelatex" if you prefer the pdfLaTeX toolchain.
%% However, pdfLaTeX will not be able to deliver the professional font experience that you
%% will get with XeLaTeX.
%%
%% _Important_: These magic comments should be on the first lines of your source file.
%%
%%%%%%%%%%%%%%%%%%%%%%%%%%%%%%%%%%%%%%%%%%%%%%%%%%%%%%%%%%%%%%%%%%%%%%%%%%%%%%%%
%%%%%%%%

%%%%%%%%%%%%%%%%%%%%%%%%%%%%%%%%%%%%%%%%%%%%%%%%%%%%%%%%%%%%%%%%%%%%%%%%%%%%%%%%
%%
%% Document class: This is a LaTeX beamer presentation.
%%
\documentclass[utf8,aspectratio=169,ngerman,english]{beamer}

\usetheme[darkmode,fancyfonts,totalframenumber,mathastext,TNF]{jku}

%%
%% Note that `nosectionpage' automatically sets `nosubsectionpage' and `nosubsubsectionpage'.
%% You could still e.g. show only subsubsection pages by using `nosectionpage,subsubsectionpage'.
%%
%% Space-efficient monospace font options (requires XeTeX):
%%  * compactmono   ... Use condensed fixed-width font everywhere.
%%  * nocompactverb ... Do not use condensed fixed-width font for verbatim and listings.
%%
%% Style-breaking options:
%%  * nojkufooter    ... Do not insert JKU/partner logos into the frame footer.
%%  * nofooter       ... Do not display a frame footer.
%%  * noimprint      ... Do not insert imprint on title pages.
%%  * frametitlecaps ... Set frame titles in capital letters (like in eariler theme versions).
%%  * nofancyfonts   ... Do not use custom TTF fonts with XeTeX / supress pdfLaTeX warning.
%%  * mac            ... Use adapted color palette for screen display on Mac.
%%
%% Experimental options:
%%  * mathastext ... Use standard document fonts (and default to sans-serif font) in math mode
%%
%% Advanced options:
%%  * nooptpackages     ... Do not load additional convenience packages (which are only there
%%                          to provide interoperability to the behavior of previous versions of
%%                          this theme but are not actually required for the current version).
%%  * logopath={<path>} ... Set the path where the theme can find its own logo resources. This
%%                          should typically be a relative path and the default is `./logos'.
%%  * fontpath={<path>} ... Set the path where the theme can find its own font resources. This
%%                          should typically be a relative path and the default is `./fonts'.
%%
%% Hint: Boolean options can be used in the forms `option' or `option=true' the enable the
%% option and `nooption' or `option=false' to disable the option.
%%
%%%%%%%%%%%%%%%%%%%%%%%%%%%%%%%%%%%%%%%%%%%%%%%%%%%%%%%%%%%%%%%%%%%%%%%%%%%%%%%%

%%%%%%%%%%%%%%%%%%%%%%%%%%%%%%%%%%%%%%%%%%%%%%%%%%%%%%%%%%%%%%%%%%%%%%%%%%%%%%%%
%%
%% This is the place where you can load additional packages. If you want to load
%% a package `booktabs', you would use the command `\usepackage{booktabs}'.
%%

\usepackage{booktabs}
\usepackage{tabularx}
%\usepackage{pifont}
%\newcommand{\cmark}{\ding{51}}
%\newcommand{\xmark}{\ding{55}}
%\newcommand{\rarr}{\ding{212}}
%\newcommand{\larr}{\raisebox{\depth}{\rotatebox{180}{\rarr}}}
\usepackage{csquotes}
\usepackage[backend=biber,citestyle=authoryear,sortcites=true,style=ACM-Reference-Format]{biblatex}

%%
%%%%%%%%%%%%%%%%%%%%%%%%%%%%%%%%%%%%%%%%%%%%%%%%%%%%%%%%%%%%%%%%%%%%%%%%%%%%%%%%

%%%%%%%%%%%%%%%%%%%%%%%%%%%%%%%%%%%%%%%%%%%%%%%%%%%%%%%%%%%%%%%%%%%%%%%%%%%%%%%%
%%
%% Set reasonable defaults for biblatex.
%%

\preto{\bibsetup}{\providecommand*{\insertbiblabel}{}}
\DeclareFieldFormat*{title}{#1}
\DeclareFieldFormat*{booktitle}{#1}
\DeclareFieldFormat*{journaltitle}{#1}
\setcounter{biburlnumpenalty}{100}
\setcounter{biburllcpenalty}{100}
\setcounter{biburlucpenalty}{100}

%% Add bibligraphy source files
\addbibresource{references.bib}

%%
%%%%%%%%%%%%%%%%%%%%%%%%%%%%%%%%%%%%%%%%%%%%%%%%%%%%%%%%%%%%%%%%%%%%%%%%%%%%%%%%


\begin{document}
\title{On energy based vector-hysteresis models}

%% Command \subtitle[short subtitle]{subtitle}: sets the presentation subtitle
\subtitle{Bachelor-Thesis presentation}

%% Command \author[short authors]{authors}: sets the presentation authors (multiple authors may be separated with \and)
\author{Franz Scharnreitner}


%% Command \institutecode{CODE}: sets the institute abbreviation/initials (used to load the institute logo file, if present)
%\institutecode{INS}

%% Command \date[short date]{date}: sets the presentation date (short date is used in the footer by default)
\date{\today}

\maketitle


\begin{frame}
\frametitle{Code overview}
 \tableofcontents
\end{frame}

\section{Theoretical foundations}
\begin{frame}
 \frametitle{Subdifferential \cite{rockafellar1997convex}}

 \begin{itemize}
    \item Let $f: \mathbb R^n \to \mathbb R$ be a continuous convex function
    \item Subdifferential: $\partial f(x) = \{p \in \mathbb R^n: f(y) \geq f(x) + \langle p, y - x \rangle \, \forall y \in \mathbb R^n\}$
    \item If $f \in C^1(\mathbb R^n,\mathbb R)$, then $\partial f(x) = \{\nabla f(x)\}$
     \item Optimality condition: $x$ is a minimum of $f$ iff \begin{align}
                                                        0 \in \partial f(x)
                                                       \end{align}
        holds.
    \item Examples:
     \begin{itemize}
      \item Subdifferential of $f(x): x \mapsto \|x\|$: $\partial f(0) = \{x : \|x\| \leq 1 \}$
      \item Let $K = \{u \in \mathbb R^3 : \|u\| \leq \chi \} $ Subdifferential of $I_K(x):$
      $$\partial I_K(x) = \begin{cases}\{0\} & \|x\| < \chi \\ \{p : p = \lambda x \,\, \forall \lambda > 0 \} & \|x\| = \chi \end{cases}$$
     \end{itemize}



 \end{itemize} 

\end{frame}


\begin{frame}
\frametitle{Lagrange function \cite{rockafellar1997convex}}
 \begin{itemize}
    \item Given the convex problem $\min_x f(x)$ s.t. $g(x) \leq 0 \,(CP)$. The Lagrange function is then defined as
    \begin{equation}
    \mathcal L_\lambda(x) = f(x) + \lambda g(x)
    \end{equation}
    For $x$ to be solution to $(CP)$ it is necessary and sufficient, that the following conditions are fulfilled:
    \begin{align}
        \lambda \geq 0, g(x) \leq 0, \lambda g(x) = 0 \\
        0 \in \partial f(x) + \lambda \partial g(x) = (\nabla \mathcal L_\lambda(x))
    \end{align}
 \end{itemize} 


\end{frame}

\begin{frame} 
 \frametitle{Legendre-Fenchel Transform \cite{doi:10.1137/1.9781611974997}}
 Let $f$ be a convex function. The Legendre-Fenchel transform, or convex conjugate is defined as
 \begin{align}
  f^*(x) =  \sup_y\{\langle x, y\rangle - f(y) \}
 \end{align}
 \begin{itemize}
  \item $f^*$ is convex
  \item If $f$ is differentiable and strictly convex, the equality $f^{**}(x) = f(x)$ holds.
  \item $f'(x) = \arg \sup_y \langle x,y \rangle - f^*(y)$
 \end{itemize} 


\end{frame}

\section{Models}
\begin{frame}
\frametitle{Unrestrained Model \cite{inproceedings}}
Can be derived from the material law $E(b,m) = \frac{\mu_0}{2} \|h\| + U(m)$, where $E$ is the internal energy
of a ferromagnetic material and $\mu_0$ is the magnetic permeability of vacuum,
$b$ the magnetic flux density, $m$ the magnetization and $h = \frac{1}{\mu_0}b - m$ the magnetic field strength,
\begin{block}{Unrestrained model}
Find
 \begin{align}
  \bar m = \arg \min_m U(m) - \langle h , m \rangle + \chi \|m - m_p\|
 \end{align}
\end{block}
\end{frame}

\begin{frame}
\frametitle{Restrained Model \cite{inproceedings}}
\begin{block}{Restrained Model}
Find
 \begin{align}
  h_r = \arg \min_{u \in K(h)} S(u) - \langle m_p , u \rangle
 \end{align}
where $K(h) = \{u \in \mathbb R^3: \|u - h\| \leq \chi \}$
\end{block}
$S$ is chosen so that
\begin{align}
 \frac{\partial S}{\partial u}(h_r) = \bar m \Leftrightarrow \frac{\partial U}{\partial m}(\bar m) = h_r
\end{align} which can be accomplished by choosing $S$ as the Legendre-Fenchel conjugate of $U$
\cite{inproceedings}


\end{frame}

\section{Equivalence}
\begin{frame}
\frametitle{Equivalence}
\begin{block}{Theorem}
 The two proposed models are equivalent in the sense that if $\bar m$ minimizes the unrestrained model, then $h_r = \frac{\partial U}{\partial m}(\bar m)$ minimizes the restrained problem and inversely if $h_r$ minimizes the restrained problem, then $\bar m = \frac{\partial S}{\partial u}(h_r)$ minimizes the unrestrained problem.
\end{block}

\end{frame}

\begin{frame}
\frametitle{Proof}

\begin{itemize}
    \item Restrained problem: $$h_r = \arg \min_{u \in K(h)} S(u) - \langle m_p , u \rangle$$
    \item Unrestrained problem: $$\bar m = \arg \min_m U(m) - \langle h , m \rangle + \chi \|m - m_p\|$$
    \item Recall: 
    \begin{itemize}
        \item Subdifferential of $f(x): x \mapsto \|x\|$: $\partial f(0) = \{x : \|x\| \leq 1 \}$
        \item Lagrange optimality condition
        \item $\frac{\partial S}{\partial u}(h_r) = \bar m \Leftrightarrow \frac{\partial U}{\partial m}(\bar m)$
    \end{itemize}
    
\end{itemize}
\end{frame}

\section{Code overview}
\begin{frame}
 \frametitle{Code overview}
  \begin{itemize}
      \item Two methods have been implemented in Julia so far.
      \begin{itemize}
          \item The restrained problem is first solved using a global newton method, the solution is checked, if it is in $K(h)$ and if not it must be on $\partial K(h)$ and newton method is run again, with a parametrization in sphere coordinates
          \item For the restrained problem the proximal gradient method is implemented
      \end{itemize}
      \item We can verify the equivalence for the test functions 
      \begin{align}
          U(m) &= m^T A m + b^T m - c \\
          S(u) &= \frac{1}{2} (m - b)^T A^{-1} (m-b) + c
      \end{align}
  \end{itemize}
\end{frame}

\begin{frame}[allowframebreaks]{References}
%\setbeamerfont{bibliography item}{size={\footnotesize}}
%\AtNextBibliography{\footnotesize}
\printbibliography
\end{frame}
\end{document}
