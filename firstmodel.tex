\documentclass[a4paper,10pt]{article}
\usepackage[utf8]{inputenc}

\usepackage{a4}
\usepackage{graphicx}
\usepackage{hyperref}
\usepackage{amsmath,amssymb}

\usepackage{biblatex}
%opening
\title{first model}
\author{Franz Scharnreitner}

\begin{document}

\maketitle

\section{Unrestrained Model}


For the energy density $E$ in a magnetic material the equation
\begin{align}
    &W = \frac{1}{2} \mu_0 h^2 + U(m) & \text{and} \label{eq:density1} \\
\end{align} ,where $h = \frac{1}{\mu_0} b - m$ and $b$ are magnetic field and magnetic induction $\mu_0$ the permiability of vacuum, holds.
Computing the dericative yields
\begin{align} \label{eq:dt}
\dot W = \mu_0 \langle h,\frac{1}{\mu_0} \dot b - \dot m \rangle + \langle \frac{\partial U}{m}(m), \dot m \rangle
\end{align}
Defining $h_r = \frac{1}{\mu_0} \frac{\partial U}{\partial m}(m)$ as the reversible part of $h = h_i + h_r$ allows us to
rewrite \eqref{eq:dt} as
\begin{align}
 \dot W = \langle h, \dot b \rangle - \langle \mu_0 h_i,\dot m \rangle
\end{align}
It is now proposed, that analogously to dry friction $\langle \mu_0 h_i,\dot m \rangle$ can be written as
\begin{align}
 \langle \mu_0 h_i,\dot m \rangle = \chi \|\dot m \|
\end{align}
This yields the

\section{Restrained Model}

For the energy density $E$ in a magnetic material the equations
\begin{align}
    &E = \frac{1}{2} \mu_0 h^2 + U(m) & \text{and} \label{eq:density} \\
    &\dot E= \langle h, \dot b \rangle - \|r\dot m\| \label{eq:dtdensity}
\end{align}
hold, with $b = \mu_0(h + m)$ being the magnetic induction. Differentiating \eqref{eq:density} and combing it with \eqref{eq:dtdensity} yields
\begin{align}
    \mu_0 \langle h,\dot h \rangle + \langle \frac{\partial U}{\partial m}(m), \dot m \rangle &= \mu_0 \langle h,\dot h + \dot m \rangle - \|r \dot m\| \Leftrightarrow \\
    \langle h - \frac{1}{\mu_0} \frac{\partial U}{\partial m}(m), \dot m \rangle &= \|\frac{r}{\mu_0} \dot m \| \label{eq:4}
\end{align}





Let $\chi = \|\frac{r}{\mu_0}\|$, $h_r = f(m)$ and $h_i = h - h_r$, then \eqref{eq:4} can be further rewritten as
\begin{align}
    \langle h_i, \dot m \rangle = \chi \|\dot m\|
\end{align}
This holds when $\dot m \in \partial I_{\tilde K}(h_i)$, where $I$ is the indicator
function and $\tilde K := \{u \in \mathbb R^3: \|u\| \leq \chi\}$,
or alternatively if $h_r \in K(h) = \{u \in R^3: \|h_r - h\| \leq \chi \}$ and $\langle \dot m, u - h_r \rangle \geq 0 \,\, \forall u \in K(h)$.
We assume $S(u)$ is the Legendre Fenchel conjugate of $\frac{1}{\mu_0}$ of $U(m)$,
which means, that $\frac{\partial S}{\partial u}(h_r) = m$. By using a pseude time stepping approch $\dot m$ can be approximated with $m - m_p$
and thus we can write the discretisised problem
as: Find $h_r \in K(h)$ so that $\langle \frac{\partial S}{\partial u}(h_r) - m_p,u - h_r \rangle \geq 0 \,\, \forall u \in K(h)$
Which is again equivalent to \begin{align}\min_{u \in K(h)} S(u) - \langle u, m_p \rangle \end{align}
\end{document}
