\documentclass[12pt]{article}
%\documentclass[11pt]{amsart}
\usepackage[utf8]{inputenc}
\usepackage{a4}
\usepackage{graphicx}
\usepackage{hyperref}
\usepackage{amsmath,amssymb}

\date{}

\setlength\parindent{0em}


\def\dx{\partial_x}
\def\dy{\partial_y}
\def\dz{\partial_z}
\def\dt{\partial_t}
\def\RR{\mathbb{R}}
\def\div{\operatorname{div}}
\def\curl{\operatorname{curl}}

\begin{document}
%\maketitle


\begin{center}
    \Large \textbf{Bachelor's-Thesis Project}
\end{center}

\bigskip 
\bigskip 
\bigskip 

\textbf{Topic:} 
On energy-based vector hysteresis models

\medskip 

\textbf{Student:} Franz Scharnreiter

\medskip 

\textbf{Supervisors:}
Michael Mandlmayr, Herbert Egger


\bigskip 

\paragraph{Scope.}

We review the two hysteresis models of \cite{Lavet} and \cite{Prigozhin}, establish their equivalence, and investigate their efficient numerical realization. 
%
Both models are based on a magneto-quasistatic consideration with magnetic energy density given by 
\begin{align} \label{eq:energy}
E(b,m) = \frac{\mu_0}{2} |h|^2 + U(m),
\end{align}
%
with $U(m)$ the energy stored in the magnetization $m$, and magnetic field strength
\begin{align} \label{eq:h}
    h = \frac{1}{\mu_0} b - m.
\end{align}
Here $b$ denotes the magnetic flux density, and $\mu_0$ the permeability of vacuum. 
%

\medskip 

\textbf{Note.}
All the following considerations concern a single material point.
%
Thus $m$, $h$, $b$ are vectors in $\RR^d$ in dimension $d=2$ or $d=3$.

\bigskip

\textbf{Approach~1.}
%
The constitutive model in \cite{Lavet} is based on the variational principle
\begin{align} \label{eq:lavet}
    m = \arg\min_m U(m) - \langle h,m\rangle + \chi |m - m_p|, 
\end{align}
with $\langle \cdot,\cdot\rangle$ denoting the Euclidean inner product, $\chi>0$ a given parameter, and $m_p \in \RR^d$ the given magnetization of the "previous time step".
%
If $U(\cdot)$ is assumed strictly convex, this uniquely determines $m$ as a function of $h$ and $m_p$ consequently allows to express $m=m(h,m_p)$ and $b=b(h,m_p)$ using \eqref{eq:h}. 
%


\bigskip 

\textbf{Approach~2.}
%
In \cite{Prigozhin}, the following alternative problem is proposed:
\begin{align} \label{eq:prigozhin}
    h_r = \arg\min_{u \in K(h)} S(u) - \langle m_p, u\rangle,
\end{align}
with $S(u)$ defined as the Legendre-Fenchel conjugate function of $\frac{1}{\mu_0} \frac{\partial U}{\partial m}(m)$, which simply means that
\begin{align} \label{eq:duality}
    \frac{\partial S}{\partial u}(h_r) = m \quad \Leftrightarrow \quad h_r = \frac{\partial U}{\partial m}(m).
\end{align}
The set $K(h)$ over which is minimized is given by 
\begin{align} \label{eq:Kh}
    K(h) = \{ u : |u-h| \le \chi\}.
\end{align}
%
Once $h_r$ is found, we can determine $m$ as a function of $h_r$ (and of $h$ and $m_p$) using \eqref{eq:duality}, and then find $b$ as a function of $h$
and $m_p$ using \eqref{eq:h}.
%
Both models are motivated by an analogy with mechanics, i.e., a model for dry friction; see~\cite{Moreau}. 
% \textbf{Numerical methods for Approach 1}
% The objective in \eqref{eq:lavet} is necessarily differentiable in every (interesting) point (eg. $m=m_p$). 
% Therefore the classical smooth optimization theory is not applicable.
% There are basically two possibilities 
% \begin{enumerate}
%     \item smooth the non-smooth part and apply a smooth optimization method.
%     \item use an optimization algorithm that does not need $C_1$.
% \end{enumerate}

% Moreover, one can use
% \begin{enumerate}
% \item first order methods (gradient based) 
% \item second order(Hessian) methods (Newton-type)
% \end{enumerate}
% The advantage of the first is that each step is very cheap, while the advantage of the second is the local (quadratic) superlinear convergence.

% The canonical first order approach to solving \eqref{eq:lavet} are definitely  proximal gradient methods.

% Relevant second order methods for solving \eqref{eq:lavet} are the semismooth Newton method and the semismooth$^*$ Newton method.
%

\newpage

\textbf{Goals of the project.}
%
The first goal of the project is to establish the equivalence of the two models, i.e., to show that for given $h$ and $m_p$, we arrive at the same $b$ in the end.
%
After doing this, we investigate and compare the numerical solution of both approaches. Some hints in this direction and appropriate test problems can be found in \cite{Lavet,Prigozhin}. 
%
Of particular interest will be to identify the most efficient and stable approach for computing $b=b(h,m_p)$ in dimension $d=3$.

\bigskip 

\textbf{Outlook.}
%
The computation of magnetic hysteresis effects is of great practical importance for the simulation and optimization of electric motors and transformers.  
%
At the moment, energy losses due to hysteresis are estimated after the simulation, which is performed without consideration of hysteresis. 
%
Results of this project might thus contribute to a further improvement of the accuracy of magnetic simulation models and their capability of predicting hysteresis losses.


\begin{thebibliography}{10}

\bibitem{Lavet}
V. François-Lavet, F. Henrotte, L. Stainier, L. Noels, and C. Geuzaine.
\newblock An energy-based variational model of ferromagnetic hysteresis for finite element computations.
\newblock {\em J. Comput. Appl. Math.}, vol. 246, pp. 243–250, 2013.

\bibitem{Prigozhin}
L. Prigozhin1, V. Sokolovsky, J. W. Barrett, and S. E. Zirka. 
\newblock On the energy-based variational model
for vector magnetic hysteresis.
\newblock {\em IEEE Trans. Magn.}, vol.~52, pp.~7301211, 2016.

\bibitem{Moreau}
J. J. Moreau. 
\newblock Application of convex analysis to the treatment of elastoplastic systems. 
\newblock In: {\em Applications of Methods of Functional Analysis to Problems in Mechanics (Lecture Notes in Mathematics)}, vol. 503, 
P. Germain and B. Nayroles, Eds.,
Springer-Verlag, Berlin, Germany, 1976, pp. 56–89.

\bibitem{kaltenbacher}
M. Kaltenbacher, K. Roppert, L. D. Domenig, and H Egger.
\newblock Comparison of Energy Based Hysteresis Models.
\newblock {\em COMPUMAG 2022}, pp. 1-4. %\href{https://doi.org/10.1109/COMPUMAG55718.2022.9827509}{doi:10.1109/COMPUMAG55718.2022.9827509}
\end{thebibliography}



\end{document}
